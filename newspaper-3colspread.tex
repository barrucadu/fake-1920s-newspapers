\documentclass{newspaper1920}

\begin{document}

\headlineMainSuper{Derelict Vessel}
\headlineMain{One Survivor and Dead Man Found Aboard}
\fancyrule
\vspace{-\baselineskip}\headlineMainSub{Tale of Desperate Battle and Deaths at Sea}
\vspace{-\baselineskip}
\begin{multicols}{3}
SYDNEY, April 18 (A.~P.)---The Morrison Co.'s freighter
\emph{Vigilant}, bound from Valparaiso, arrived this morning at its
wharf in Darling Harbour, having in tow the battled and disabled but
heavily armed steam yacht \emph{Alert} of Dunedin, N.~Z., which was
sighted April 12th.

The \emph{Vigilant} left Valparaiso March 25th, and on April 2d was
driven considerably south of her course by exceptionally heavy storms
and monster waves.  On April 12th the derelict was sighted; and though
apparently deserted, was found upon boarding to contain one survivor
in a half-delirious condition and one man who had evidently been dead
for more than a week.

The living man was clutching a horrible stone idol of unknown origin,
about a foot in height, regarding whose nature authorities at Sydney
University, the Royal Society, and the Museum in College Street all
profess complete bafflement, and which the survivor says he found in
the cabin of the yacht, in a small carved shrine of common pattern.

This man, after he had recovered his senses, told an exceedingly
strange story of piracy and slaughter.  He is Gustaf Johansen, a
Norwegian of some intelligence, and had been second mate of the
two-masted schooner \emph{Emma} of Auckland, which sailed for Callao
February 20th, with a complement of eleven men.

The \emph{Emma}, he says, was delayed and thrown widely south of her
course by the great storm of March 1st, and on March 22d encountered
the \emph{Alert}, manned by a queer and evil-looking crew of Kanakas
and half-castes.  Being ordered peremptorily to turn back,
Capt. Collins refused; whereupon the strange crew began to fire
savagely and without warning upon the schooner with a peculiarly
heavy battery of brass cannon forming part of the yacht's equipment.

Cable advices from Dunedin report that the \emph{Alert} was well known
there as an island trader, and bore an evil reputation along the
waterfront.  It was owned by a curious group who attracted no little
attention; and it had set sail in great haste just after the storm and
earth tremors of March 1st.

\headlineMainInner{Admiralty to investigate}

The admiralty will institute an inquiry on the whole matter beginning
tomorrow, at which every effort will be made to induce Johansen to
speak more freely than he has done hitherto.

\vfill
\headlineMainInner{\scshape Continued on page 2}

\end{multicols}

\vspace{-\baselineskip}\fullrule
\vspace{-\baselineskip}

\begin{multicols}{3}

\headlineMain{Di Robilant's Aide Dies of Jungle Fever}
\headlineMainSub{Italian Aviator Recovering}

SAO PAULO (Reuters)---A telegram received at Sao Paulo late today
advised the Italian Consul that Mauranta Quarenta, mechanic of Count
di Robilant's plane, died at San José this morning from injuries and
exhaustion.  Count di Robilant, occupying the next bed, when in formed
of Signor Quarenta's death shortly before noon, at first refused to
believe it.  Later he developed a high fever, causing considerable
alarm.  The hospital doctors now fear di Robilant may be suffering
from a tropical fever contracted during his eighteen days of
wandering, after the crash of his plane.

The Italian Consul at Sao Paulo informed Signor Quarenta's family,
living at Rio de Janeiro, of his death and the Italian Government will
transport the family to Italy.

Count di Robilant is unable to see visitors and the hospital doctors
here advised the Vice Consul at Sao Paulo that he will be transported
tomorrow morning to Assis, where better medical assistance is
obtainable.  He is believed to be out of danger, but requires special
treatment.

Signor Quarenta died as the result of yellow fever contracted on the
tortuous walk through dense forests, it is reported.  Count di
Robilant, although also affected, is expected to regain his health
soon.

\fullrule

\headlineMain{Murderers Hanged}

NAIROBI (Reuters)---Five Nandi tribesmen, convicted ringleaders of
the vicious Carlyle Expedition massacre, were executed this morning
after a short, expertly-conducted trial.

To the end, the tribesmen steadfastly refused to reveal why they had
slaughtered Mr. Carlyle and his companions.  Mr. Harvis, acting for
the Colony, cleverly implied throughout the trial that the massacre
was racial in motivation, and that the fair-skinned victims were
subject to the most savage treatment, preventing all but the most
preliminary identification of the remains.

Miss Erica Carlyle, defeated in her efforts to rescue her brother,
left several weeks ago, but is surely comforted now the triumph of
justice.

\headlineMain{Foreign Cables}

\headlineSmall{Artist Divorced}

PARIS, April 16 (A.~P.)---Mrs.~Lucile Wolfe Burton was granted a
divorce today from William O.\ Burton, American portrait painter, now
living in Paris.

\halfrule

\headlineSmall{Gold in Glacier}

STOCKHOLM, April 18 (A.~P.)---Norwegians may seek unknown land, but
some Swede engineers have found rich gold, silver and copper deposits
in an arctic glacier, using new electrical prospecting apparatus.

\halfrule

\headlineSmall{Wrecked Train}

MEXICO CITY, April 18 (A.~P.)---The train which was wrecked near
Pachuca was carrying troops from Pahuca to Ixmiquilpan.  Ten men were
reported killed and 30 injured.  A military relief train was sent to
the scene.

\halfrule

\headlineSmall{Prince Drove Truck}

LONDON, April 17 (A.~P.)---The prince of Wales is becoming a regular
Harun Al Raschid.  It develops that during the strike he drove a motor
truck and helped distribute milk to the poor.

\halfrule

\headlineSmall{100 Killed in Battle}

TANGIER, Morocco, April 18 (A.~P.)---More than a hundred members of
the Spanish foreign legion are reported to have been killed and a
large number wounded in fighting at Rio Martin, near Tetuan.  Another
party numbering 360 persons are reported to have been surrounded in a
ravine by the tribesmen, all being captured or killed.

\halfrule

\headlineSmall{Ignores Rebels}

MANAGUCA, Nigaragua, April 17 (A.~P.)---The revolutionary governor at
Bluefields has called upon W.~J.~Crampton, American collector of
customs there, to deliver the customs revenue to him.  The collector
refused and has requested protection for himself and his staff against
the rebels.

\columnbreak

\headline{Brisbane Denies Marriage Rumors}

Mr.~Luther Brisbane, proprietor of Arkham's own Brisbane Brick Works,
today said there was ``absolutely nothing'' in rumors that he was soon
to be married to Mrs.~Adolph Spreckles of San Francisco, widow of the
nationally known sugar magnate.

\halfrule

\headline{New Guard Needed for Leper House}

The District Health Department wants another guard for Willard
Centlivre and Charles H. Young, held at the leper house, until the
Public Health Service can take them over.

Someone is needed to relieve the guard now on duty at the leper house.
The department needs a man who can give approximately eight hours a
day, one day a week, for \$3 a day, preferable a man who has regular
employment at night.

\halfrule

\headline{Flashes Badge at Hotel, Pays \$5 Fine}

Flashing a badge that looked like that worn by policemen, Roman
Struck, 39 years old, 530 Powder Mill st., threatened the owners of a
hotel at Gedney and Armitage st.\ with arrest.  The sight of two
policemen, who were called, induced Struck to step into his more
accustomed role of a civilian.  When questioned by police as to why he
made the impersonation, he replied, ``I thought it would be fun.''

In district court today he was fined \$5 on charge of being drunk.

\halfrule

\headline{Successful Entertainment at Vaughnville~School}

One of the enjoyable social features of the Vaghnville community was
the entertainment given by the school at the new building.  There was
not a single hitch in the execution of the program from the beginning
to the finis.  Each child did his part remarkably well, especially
those beginners, who successfully dramatized the nursery rhyme of
Lucky Locket.  The play, ``Mother Goose Up to Date,'' was a great
success.

\end{multicols}

\end{document}
